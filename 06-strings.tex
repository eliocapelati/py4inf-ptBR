% LaTeX source for ``Python for Informatics: Exploring Information''
% Copyright (c)  2010-  Charles R. Severance, All Rights Reserved

\chapter{Strings}
\label{strings}


\section{A string is a sequence}
\index{sequence}
\index{character}
\index{bracket operator}
\index{operator!bracket}

A string is a {\bf sequence} of characters.  
You can access the characters one at a time with the
bracket operator:

\beforeverb
\begin{verbatim}
>>> fruit = 'banana'
>>> letter = fruit[1]
\end{verbatim}
\afterverb
%
\index{index}
The second statement extracts the character at index position 1 from the 
{\tt fruit} variable and assigns it to the {\tt letter} variable.  

The expression in brackets is called an {\bf index}.  
The index indicates which character in the sequence you
want (hence the name).

But you might not get what you expect:

\beforeverb
\begin{verbatim}
>>> print letter
a
\end{verbatim}
\afterverb
%
For most people, the first letter of \verb"'banana'" is {\tt b}, not
{\tt a}.  But in Python, the index is an offset from the
beginning of the string, and the offset of the first letter is zero.

\beforeverb
\begin{verbatim}
>>> letter = fruit[0]
>>> print letter
b
\end{verbatim}
\afterverb
%
So {\tt b} is the 0th letter (``zero-eth'') of \verb"'banana'", {\tt a}
is the 1th letter (``one-eth''), and {\tt n} is the 2th (``two-eth'')
letter.

\beforefig
\centerline{\includegraphics[height=0.50in]{figs2/string.eps}}
\afterfig

\index{index!starting at zero}
\index{zero, index starting at}

You can use any expression, including variables and operators, as an
index, but the value of the index has to be an integer.  Otherwise you
get:

\index{index}
\index{exception!TypeError}
\index{TypeError}

\beforeverb
\begin{verbatim}
>>> letter = fruit[1.5]
TypeError: string indices must be integers
\end{verbatim}
\afterverb
%

\section{Getting the length of a string using {\tt len}}

\index{len function}
\index{function!len}

{\tt len} is a built-in function that returns the number of characters
in a string:

\beforeverb
\begin{verbatim}
>>> fruit = 'banana'
>>> len(fruit)
6
\end{verbatim}
\afterverb
%
To get the last letter of a string, you might be tempted to try something
like this:

\index{exception!IndexError}
\index{IndexError}

\beforeverb
\begin{verbatim}
>>> length = len(fruit)
>>> last = fruit[length]
IndexError: string index out of range
\end{verbatim}
\afterverb
%
The reason for the {\tt IndexError} is that there is no letter in {\tt
'banana'} with the index 6.  Since we started counting at zero, the
six letters are numbered 0 to 5.  To get the last character, you have
to subtract 1 from {\tt length}:

\beforeverb
\begin{verbatim}
>>> last = fruit[length-1]
>>> print last
a
\end{verbatim}
\afterverb
%
Alternatively, you can use negative indices, which count backward from
the end of the string.  The expression {\tt fruit[-1]} yields the last
letter, {\tt fruit[-2]} yields the second to last, and so on.

\index{index!negative}
\index{negative index}


\section{Traversal through a string with a loop}
\label{for}

\index{traversal}
\index{loop!traversal}
\index{for loop}
\index{loop!for}
\index{statement!for}
\index{traversal}

A lot of computations involve processing a string one character at a
time.  Often they start at the beginning, select each character in
turn, do something to it, and continue until the end.  This pattern of
processing is called a {\bf traversal}.  One way to write a traversal
is with a {\tt while} loop:

\beforeverb
\begin{verbatim}
index = 0
while index < len(fruit):
    letter = fruit[index]
    print letter
    index = index + 1
\end{verbatim}
\afterverb
%
This loop traverses the string and displays each letter on a line by
itself.  The loop condition is {\tt index < len(fruit)}, so
when {\tt index} is equal to the length of the string, the
condition is false, and the body of the loop is not executed.  The
last character accessed is the one with the index {\tt len(fruit)-1},
which is the last character in the string.

\begin{ex}
Write a {\tt while} loop that starts at the last character in the string
and works its way backwards to the first character in the string, 
printing each letter on a separate line, except backwards.
\end{ex}

Another way to write a traversal is with a {\tt for} loop:

\beforeverb
\begin{verbatim}
for char in fruit:
    print char
\end{verbatim}
\afterverb
%
Each time through the loop, the next character in the string is assigned
to the variable {\tt char}.  The loop continues until no characters are
left.


\section{String slices}
\label{slice}

\index{slice operator}
\index{operator!slice}
\index{index!slice}
\index{string!slice}
\index{slice!string}

A segment of a string is called a {\bf slice}.  Selecting a slice is
similar to selecting a character:

\beforeverb
\begin{verbatim}
>>> s = 'Monty Python'
>>> print s[0:5]
Monty
>>> print s[6:12]
Python
\end{verbatim}
\afterverb
%
The operator {\tt [n:m]} returns the part of the string from the 
``n-eth'' character to the ``m-eth'' character, including the first but
excluding the last.  

If you omit the first index (before the colon), the slice starts at
the beginning of the string.  If you omit the second index, the slice
goes to the end of the string:

\beforeverb
\begin{verbatim}
>>> fruit = 'banana'
>>> fruit[:3]
'ban'
>>> fruit[3:]
'ana'
\end{verbatim}
\afterverb
%
If the first index is greater than or equal to the second the result
is an {\bf empty string}, represented by two quotation marks:

\index{quotation mark}

\beforeverb
\begin{verbatim}
>>> fruit = 'banana'
>>> fruit[3:3]
''
\end{verbatim}
\afterverb
%
An empty string contains no characters and has length 0, but other
than that, it is the same as any other string.

\begin{ex}
Given that {\tt fruit} is a string, what does
{\tt fruit[:]} mean?

\index{copy!slice}
\index{slice!copy}


\end{ex}


\section{Strings are immutable}
\index{mutability}
\index{immutability}
\index{string!immutable}

It is tempting to use the {\tt []} operator on the left side of an
assignment, with the intention of changing a character in a string.
For example:

\index{TypeError}
\index{exception!TypeError}

\beforeverb
\begin{verbatim}
>>> greeting = 'Hello, world!'
>>> greeting[0] = 'J'
TypeError: object does not support item assignment
\end{verbatim}
\afterverb
%
The ``object'' in this case is the string and the ``item'' is
the character you tried to assign.  For now, an {\bf object} is
the same thing as a value, but we will refine that definition
later.  An {\bf item} is one of the values in a sequence.

\index{object}
\index{item assignment}
\index{assignment!item}
\index{immutability}

The reason for the error is that
strings are {\bf immutable}, which means you can't change an
existing string.  The best you can do is create a new string
that is a variation on the original:

\beforeverb
\begin{verbatim}
>>> greeting = 'Hello, world!'
>>> new_greeting = 'J' + greeting[1:]
>>> print new_greeting
Jello, world!
\end{verbatim}
\afterverb
%
This example concatenates a new first letter onto
a slice of {\tt greeting}.  It has no effect on
the original string.

\index{concatenation}

\section{Looping and counting}
\label{counter}

\index{counter}
\index{counting and looping}
\index{looping and counting}
\index{looping!with strings}

The following program counts the number of times the letter {\tt a}
appears in a string:

\beforeverb
\begin{verbatim}
word = 'banana'
count = 0
for letter in word:
    if letter == 'a':
        count = count + 1
print count
\end{verbatim}
\afterverb
%
This program demonstrates another pattern of computation called a {\bf
counter}.  The variable {\tt count} is initialized to 0 and then
incremented each time an {\tt a} is found.
When the loop exits, {\tt count}
contains the result---the total number of {\tt a}'s.

\begin{ex}
\index{encapsulation}

Encapsulate this code in a function named {\tt
count}, and generalize it so that it accepts the string and the
letter as arguments.
\end{ex}

\section{The {\tt in} operator}
\label{inboth}

\index{in operator}
\index{operator!in}
\index{boolean operator}
\index{operator!boolean}

The word {\tt in} is a boolean operator that takes two strings and
returns {\tt True} if the first appears as a substring in the second:

\beforeverb
\begin{verbatim}
>>> 'a' in 'banana'
True
>>> 'seed' in 'banana'
False
\end{verbatim}
\afterverb
%

\section{String comparison}

\index{string!comparison}
\index{comparison!string}

The comparison operators work on strings.  To see if two strings are equal:

\beforeverb
\begin{verbatim}
if word == 'banana':
    print  'All right, bananas.'
\end{verbatim}
\afterverb
%
Other comparison operations are useful for putting words in alphabetical
order:

\beforeverb
\begin{verbatim}
if word < 'banana':
    print 'Your word,' + word + ', comes before banana.'
elif word > 'banana':
    print 'Your word,' + word + ', comes after banana.'
else:
    print 'All right, bananas.'
\end{verbatim}
\afterverb
%
Python does not handle uppercase and lowercase letters the same way
that people do.  All the uppercase letters come before all the
lowercase letters, so:

\beforeverb
\begin{verbatim}
Your word, Pineapple, comes before banana.
\end{verbatim}
\afterverb
%
A common way to address this problem is to convert strings to a
standard format, such as all lowercase, before performing the
comparison.  Keep that in mind in case you have to defend yourself
against a man armed with a Pineapple.


\section{{\tt string} methods}

Strings are an example of Python {\bf objects}.  An object contains
both data (the actual string itself) and {\bf methods}, which
are effectively functions that are built into the object and 
are available to any {\bf instance} of the object.

Python has a function called {\tt dir} which lists the methods available
for an object.  The {\tt type} function shows the type of an object 
and the {\tt dir} function shows the available methods.

\beforeverb
\begin{verbatim}
>>> stuff = 'Hello world'
>>> type(stuff)
<type 'str'>
>>> dir(stuff)
['capitalize', 'center', 'count', 'decode', 'encode', 
'endswith', 'expandtabs', 'find', 'format', 'index', 
'isalnum', 'isalpha', 'isdigit', 'islower', 'isspace', 
'istitle', 'isupper', 'join', 'ljust', 'lower', 'lstrip', 
'partition', 'replace', 'rfind', 'rindex', 'rjust', 
'rpartition', 'rsplit', 'rstrip', 'split', 'splitlines', 
'startswith', 'strip', 'swapcase', 'title', 'translate', 
'upper', 'zfill']
>>> help(str.capitalize)
Help on method_descriptor:

capitalize(...)
    S.capitalize() -> string
    
    Return a copy of the string S with only its first character
    capitalized.
>>>
\end{verbatim}
\afterverb
%

While the {\tt dir} function lists the methods, and you 
can use {\tt help} to get some simple documentation on a method, 
a better source of documentation for string methods would be
\url{https://docs.python.org/2/library/stdtypes.html#string-methods}.

Calling a {\bf method} is similar to calling a function---it 
takes arguments and returns a value---but the syntax is different.
We call a method by appending the method name to the variable name
using the period as a delimiter.

For example, the
method {\tt upper} takes a string and returns a new string with
all uppercase letters:

\index{method}
\index{string!method}

Instead of the function syntax {\tt upper(word)}, it uses
the method syntax {\tt word.upper()}.

\index{dot notation}

\beforeverb
\begin{verbatim}
>>> word = 'banana'
>>> new_word = word.upper()
>>> print new_word
BANANA
\end{verbatim}
\afterverb
%
This form of dot notation specifies the name of the method, {\tt
upper}, and the name of the string to apply the method to, {\tt
word}.  The empty parentheses indicate that this method takes no
argument.

\index{parentheses!empty}

A method call is called an {\bf invocation}; in this case, we would
say that we are invoking {\tt upper} on the {\tt word}.

\index{invocation}

For example, there is a string method named {\tt find} that
searches for the position of one string within another:

\beforeverb
\begin{verbatim}
>>> word = 'banana'
>>> index = word.find('a')
>>> print index
1
\end{verbatim}
\afterverb
%
In this example, we invoke {\tt find} on {\tt word} and pass
the letter we are looking for as a parameter.

The {\tt find} method can find substrings as well as characters:

\beforeverb
\begin{verbatim}
>>> word.find('na')
2
\end{verbatim}
\afterverb
%
It can take as a second argument the index where it should start:

\index{optional argument}
\index{argument!optional}

\beforeverb
\begin{verbatim}
>>> word.find('na', 3)
4
\end{verbatim}
\afterverb
%
One common task is to remove white space (spaces, tabs, or newlines) from
the beginning and end of a string using the {\tt strip} method:

\beforeverb
\begin{verbatim}
>>> line = '  Here we go  '
>>> line.strip()
'Here we go'
\end{verbatim}
\afterverb
%
Some methods such as {\bf startswith} return boolean values.

\beforeverb
\begin{verbatim}
>>> line = 'Please have a nice day'
>>> line.startswith('Please')
True
>>> line.startswith('p')
False
\end{verbatim}
\afterverb
%
You will note that {\tt startswith} requires case to match, so sometimes
we take a line and map it all to lowercase before we do any checking
using the {\tt lower} method.

\beforeverb
\begin{verbatim}
>>> line = 'Please have a nice day'
>>> line.startswith('p')
False
>>> line.lower()
'please have a nice day'
>>> line.lower().startswith('p')
True
\end{verbatim}
\afterverb
%
In the last example, the method {\tt lower} is called
and then we use {\tt startswith}
to see if the resulting lowercase string
starts with the letter ``p''.  As long as we are careful
with the order, we can make multiple method calls in a
single expression.

\begin{ex}
\index{count method}
\index{method!count}

There is a string method called {\tt count} that is similar
to the function in the previous exercise.  Read the documentation
of this method at
\url{https://docs.python.org/2/library/stdtypes.html#string-methods}
and write an invocation that counts the number of times the 
letter a  occurs
in \verb"'banana'".
\end{ex}

\section{Parsing strings}

Often, we want to look into a string and find a substring.  For example
if we were presented a series of lines formatted as follows:

\beforeverb
\begin{alltt}
From stephen.marquard@{\bf uct.ac.za} Sat Jan  5 09:14:16 2008
\end{alltt}
\afterverb

and we wanted to pull out only the second half of the address (i.e.,
{\tt uct.ac.za}) from each line, we can do this by using the {\tt find}
method and string slicing.   

First, we will find the position of the at-sign in the string.  Then we will
find the position of the first space \emph{after} the at-sign.  And then we
will use string slicing to extract the portion of the string which we 
are looking for.

\beforeverb
\begin{verbatim}
>>> data = 'From stephen.marquard@uct.ac.za Sat Jan  5 09:14:16 2008'
>>> atpos = data.find('@')
>>> print atpos
21
>>> sppos = data.find(' ',atpos)
>>> print sppos
31
>>> host = data[atpos+1:sppos]
>>> print host
uct.ac.za
>>> 
\end{verbatim}
\afterverb
%
We use a version of the {\tt find} method which allows us to specify
a position in the string where we want {\tt find} to start looking.
When we slice, we extract the characters 
from ``one beyond the at-sign through up to \emph{but not including} the 
space character''.  

The documentation for the {\tt find} method is available at
\url{https://docs.python.org/2/library/stdtypes.html#string-methods}.

\section{Format operator}

\index{format operator}
\index{operator!format}

The {\bf format operator}, {\tt \%}
allows us to construct strings, replacing parts of the strings
with the data stored in variables.
When applied to integers, {\tt \%} is the modulus operator.  But
when the first operand is a string, {\tt \%} is the format operator.

\index{format string}

The first operand is the {\bf format string}, which contains
one or more {\bf format sequences} that specify how
the second operand is formatted.  The result is a string.

\index{format sequence}

For example, the format sequence \verb"'%d'" means that
the second operand should be formatted as an
integer ({\tt d} stands for ``decimal''):

\beforeverb
\begin{verbatim}
>>> camels = 42
>>> '%d' % camels
'42'
\end{verbatim}
\afterverb
%
The result is the string \verb"'42'", which is not to be confused
with the integer value {\tt 42}.

A format sequence can appear anywhere in the string,
so you can embed a value in a sentence:

\beforeverb
\begin{verbatim}
>>> camels = 42
>>> 'I have spotted %d camels.' % camels
'I have spotted 42 camels.'
\end{verbatim}
\afterverb
%
If there is more than one format sequence in the string,
the second argument has to be a tuple\footnote{A tuple is a
sequence of comma-separated values inside a pair of brackets.
We will cover tuples in Chapter 10}.  Each format sequence is
matched with an element of the tuple, in order.

The following example uses \verb"'%d'" to format an integer,
\verb"'%g'" to format
a floating-point number (don't ask why), and \verb"'%s'" to format
a string:

\beforeverb
\begin{verbatim}
>>> 'In %d years I have spotted %g %s.' % (3, 0.1, 'camels')
'In 3 years I have spotted 0.1 camels.'
\end{verbatim}
\afterverb
%
The number of elements in the tuple must match the number
of format sequences in the string.  The types of the
elements also must match the format sequences:

\index{exception!TypeError}
\index{TypeError}

\beforeverb
\begin{verbatim}
>>> '%d %d %d' % (1, 2)
TypeError: not enough arguments for format string
>>> '%d' % 'dollars'
TypeError: illegal argument type for built-in operation
\end{verbatim}
\afterverb
%
In the first example, there aren't enough elements; in the
second, the element is the wrong type.

The format operator is powerful, but it can be difficult to use.  You
can read more about it at
\url{https://docs.python.org/2/library/stdtypes.html#string-formatting}.

% You can specify the number of digits as part of the format sequence.
% For example, the sequence \verb"'%8.2f'"
% formats a floating-point number to be 8 characters long, with
% 2 digits after the decimal point:

% \beforeverb
% \begin{verbatim}
% >>> '%8.2f' % 3.14159
% '    3.14'
% \end{verbatim}
% \afterverb
% %
% The result takes up eight spaces with two
% digits after the decimal point.  


\section{Debugging}
\index{debugging}

A skill that you should cultivate as you program is always
asking yourself, ``What could go wrong here?'' or alternatively,
``What crazy thing might our user do to crash our (seemingly) 
perfect program?''

For example, look at the program which we used to demonstrate
the {\tt while} loop in the chapter on iteration:

\beforeverb
\begin{verbatim}
while True:
    line = raw_input('> ')
    if line[0] == '#' :
        continue
    if line == 'done':
        break
    print line

print 'Done!'
\end{verbatim}
\afterverb
%
Look what happens when the user enters an empty line of input:

\beforeverb
\begin{verbatim}
> hello there
hello there
> # don't print this
> print this!
print this!
> 
Traceback (most recent call last):
  File "copytildone.py", line 3, in <module>
    if line[0] == '#' :
\end{verbatim}
\afterverb
%
The code works fine until it is presented an empty line.  Then
there is no zero-th character, so we get a traceback.  There are two
solutions to this to make line three ``safe'' even if the line is 
empty.

One possibility is to simply use the {\tt startswith} method
which returns {\tt False} if the string is empty.

\beforeverb
\begin{verbatim}
    if line.startswith('#') :
\end{verbatim}
\afterverb
%
\index{guardian pattern}
\index{pattern!guardian}
Another way is to safely write the {\tt if} statement using the {\bf guardian}
pattern and make sure the second logical expression is evaluated 
only where there is at least one character in the string.:

\beforeverb
\begin{verbatim}
    if len(line) > 0 and line[0] == '#' :
\end{verbatim}
\afterverb
%

\section{Glossary}

\begin{description}

\item[counter:] A variable used to count something, usually initialized
to zero and then incremented.
\index{counter}

\item[empty string:] A string with no characters and length 0, represented
by two quotation marks.
\index{empty string}

\item[format operator:] An operator, {\tt \%}, that takes a format
string and a tuple and generates a string that includes
the elements of the tuple formatted as specified by the format string.
\index{format operator}
\index{operator!format}

\item[format sequence:] A sequence of characters in a format string,
like {\tt \%d}, that specifies how a value should be formatted.
\index{format sequence}

\item[format string:] A string, used with the format operator, that
contains format sequences.
\index{format string}

\item[flag:] A boolean variable used to indicate whether a condition
is true.
\index{flag}

\item[invocation:] A statement that calls a method.
\index{invocation}

\item[immutable:] The property of a sequence whose items cannot
be assigned.
\index{immutability}

\item[index:] An integer value used to select an item in
a sequence, such as a character in a string.
\index{index}

\item[item:] One of the values in a sequence.
\index{item}

\item[method:] A function that is associated with an object and called
using dot notation.
\index{method}

\item[object:] Something a variable can refer to.  For now,
you can use ``object'' and ``value'' interchangeably.
\index{object}

\item[search:] A pattern of traversal that stops
when it finds what it is looking for.
\index{search pattern}
\index{pattern!search}

\item[sequence:] An ordered set; that is, a set of
values where each value is identified by an integer index.
\index{sequence}

\item[slice:] A part of a string specified by a range of indices.
\index{slice}

\item[traverse:] To iterate through the items in a sequence,
performing a similar operation on each.
\index{traversal}

\end{description}


\section{Exercises}

\begin{ex}
Take the following Python code that stores a string:`

\beforeverb
\begin{alltt}
str = 'X-DSPAM-Confidence: {\bf 0.8475}'
\end{alltt}
\afterverb

Use {\tt find} and string slicing to extract the portion
of the string after the colon character and then use the 
{\tt float} function to convert the extracted string 
into a floating point number.

\end{ex}


\begin{ex}
\index{string method}
\index{method!string}

Read the documentation of the string methods at
\url{https://docs.python.org/2/library/stdtypes.html#string-methods}.
You might want to experiment with some of them to make sure
you understand how they work.  {\tt strip} and
{\tt replace} are particularly useful.

The documentation uses a syntax that might be confusing.
For example, in \verb"find(sub[, start[, end]])", the brackets
indicate optional arguments.  So {\tt sub} is required, but
{\tt start} is optional, and if you include {\tt start},
then {\tt end} is optional.
\end{ex}

